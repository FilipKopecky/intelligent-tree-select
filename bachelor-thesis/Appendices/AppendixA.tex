% Appendix A

\chapter{Intelligent Tree Select component API} % Main appendix title

\label{AppendixA} % For referencing this appendix elsewhere, use \ref{AppendixA}

All available select props are described here: \url{https://github.com/JedWatson/react-select#select-props} and here: \url{https://github.com/bvaughn/react-virtualized-select/#react-virtualized-select-props}. Additional parameters used by VirtualizedTreeSelect component are described in this table:


\begin{center}
    \begin{longtable}{ | l | l | l | p {6cm} |}
    \caption{Intelligent tree data management component API}\\ \hline
    \multicolumn{1}{|c|}{\textbf{Property}} & 
    \multicolumn{1}{c|}{\textbf{Type}} & 
    \multicolumn{1}{c|}{\textbf{Default value}} & 
    \multicolumn{1}{c|}{\textbf{Description}} \\ \hline 
	\endfirsthead
    \multicolumn{4}{c}%
	{\tablename\ \thetable\ -- \textit{Continued from previous page}} \\
	\hline
	\textbf{Property} & \textbf{Type} & \textbf{Default value} & \textbf{Description} \\
    \endhead
    \multicolumn{4}{r}{\textit{Continued on next page}} \\
	\endfoot
    \hline
	\endlastfoot
    
    childrenKey & PropTypes.string & 'children' & Attribute of option that contains the values (ID) of the children options \\ \hline
    valueKey & PropTypes.string & 'value' & Attribute of option that contains the values of the option \\ \hline
    labelKey & PropTypes.string & 'label' & Attribute of option that contains option label\\ \hline
    labelValue & PropTypes.func & null &  Function that is called only if option[labelKey] is an object not an string. This function get 												option[labelKey] as an parameter and must return a string value. This is useful e.g. if your data are 											multi-language\\ \hline
    
    simpleTreeData & PropTypes.bool & true & Dataset is in simplified format \\ \hline
    expanded & PropTypes.bool & true & Attribute if all options are expanded by default or not \\ \hline
    renderAsTree & PropTypes.bool & true & Attribute if options should be rendered as a tree. If false options are rendered 
    										normally as for default select \\ \hline
    displayInfoOnHover & PropTypes.bool & false & Display tool-tip with additional information \\ \hline      
    displayState & PropTypes.bool & false & Should display state of the option (local, external, new, merged) \\ \hline   
    
    optionRenderer & PropTypes.func & null & Custom way to render options (see below) \\ \hline
    filterOptions & PropTypes.func & null & Custom way to filter options (see below) \\ \hline
    onOptionCreate & PropTypes.func & null & Callback on creating a new option \\ \hline
    
    options & PropTypes.array & [] & Array of default options \\ \hline
    providers & PropTypes.array & [] & Array of provider objects\\ \hline
    
    \end{longtable}
\end{center}

\noindent \textbf{\Large{Provider object structure}}

\begin{center}
\begin{longtable}{ | l | l | l | p {6cm} |}
    \caption{API of the provider object}\\ \hline
    \multicolumn{1}{|c|}{\textbf{Property}} & 
    \multicolumn{1}{c|}{\textbf{Type}} & 
    \multicolumn{1}{c|}{\textbf{Default value}} & 
    \multicolumn{1}{c|}{\textbf{Description}} \\ \hline 
	\endfirsthead
    \multicolumn{4}{c}%
	{\tablename\ \thetable\ -- \textit{Continued from previous page}} \\
	\hline
	\textbf{Property} & \textbf{Type} & \textbf{Default value} & \textbf{Description} \\
    \endhead
    \multicolumn{4}{r}{\textit{Continued on next page}} \\
	\endfoot
    \hline
	\endlastfoot
    
    name & PropTypes.string & (required) & Unique identification of each provider \\ \hline
    response & PropTypes.func & (required) & Function that return data. This function get one string parameter that is equal to current input  \\ 											\hline
    toJsonArr & PropTypes.func & null & Function that is called to convert providers response to JSON array. This function is called only when 											response is not an JSON object \\ \hline

	
    childrenKey & PropTypes.string & 'children' & Attribute of option that contains the values (ID) of the children options \\ \hline
    valueKey & PropTypes.string & 'value' & Attribute of option that contains the values of the option \\ \hline
    labelKey & PropTypes.string & 'label' & Attribute of option that contains option label\\ \hline
    labelValue & PropTypes.func & null &  Function that is called only if option[labelKey] is an object not an string. This function get 												option[labelKey] as an parameter and must return a string value. This is useful e.g. if your data are 											multi-language\\ \hline
    
    simpleTreeData & PropTypes.bool & true & Dataset is in simplified format \\ \hline
    
	\end{longtable}
\end{center}

\pagebreak
\noindent \textbf{\Large{Custom option renderer}}


\begin{center}
    \begin{longtable}{ | l | p {4cm} | p {6cm} |}
    \caption{Option renderer method API}\\ \hline
    \multicolumn{1}{|c|}{\textbf{Property}} & 
    \multicolumn{1}{c|}{\textbf{Type}} & 
    \multicolumn{1}{c|}{\textbf{Description}}
	\endfirsthead
    \multicolumn{3}{c}%
	{\tablename\ \thetable\ -- \textit{Continued from previous page}} \\
	\hline
	\textbf{Property} & \textbf{Type} & \textbf{Description} \\
    \endhead
    \multicolumn{3}{r}{\textit{Continued on next page}} \\
	\endfoot
    \hline
	\endlastfoot
    \hline
    
    focusedOption & PropTypes.object & The option currently-focused in the drop-down. Use this property to determine if your rendered option should be highlighted or styled differently. \\ \hline
    focusedOptionIndex & PropTypes.number & Index of the currently-focused option. \\ \hline
    focusOption	& PropTypes.func & Callback to update the focused option; for example, you may want to call this function on mouse-over. \\ \hline
    labelKey & PropTypes.string & The attribute of option that contains the display text. \\ \hline
    option & PropTypes.object & The option to be rendered. \\ \hline
    options & PropTypes.arrayOf (PropTypes.object) & Array of options (objects) contained in the select menu. \\ \hline
    selectValue & PropTypes.func & Callback to update the selected values; for example, you may want to call this function on click. \\ \hline
    style & PropTypes.object & Styles that must be passed to the rendered option. These styles are specifying the position of each option (required for correct option displaying in the drop-down). \\ \hline
    valueArray & PropTypes.arrayOf (PropTypes.object) & An array of the currently-selected options. Use this property to determine if your rendered option should be highlighted or styled differently. \\ \hline
    valueKey & PropTypes.string & Attribute of option that contains the value. \\ \hline
    onToggleClick & PropTypes.func & Callback to event for clicking on expand button \\ \hline
    childrenKey & PropTypes.string & Attribute of option that contains the values of children options \\ \hline
    
    \end{longtable}
\end{center}


\noindent \textbf{\Large{Custom filter options}}

By default, a component uses a custom function for filtering the options. I don’t recommend overriding this method unless you know what you are doing. For more details, you can look at \url{https://github.com/JedWatson/react-select#advanced-filters}