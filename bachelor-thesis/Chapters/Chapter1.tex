% Chapter Template

\chapter{Introduction} % Main chapter title

\label{Chapter1} % For referencing the chapter elsewhere, use \ref{Chapter1} 

%----------------------------------------------------------------------------------------
%	SECTION 1
%----------------------------------------------------------------------------------------

\section{Overview}

Data, pieces of information that are measured, collected, analyzed and used for different kinds of things. Such data are distinguished based on their structure - unstructured (e.g., a list of strings) or structured (e.g., a graph), on their type - specific or general, and so on. Structured data are often represented as a graph. Human intuition allows us to understand these graph connections with no problem, but computers do not have that intuition. So the data context must be defined to enable the computer to understand these connections between the data and the meaning of that data. This problem is one of the main reasons why people created a concept of the Web Ontology Language \parencite{OWL}. However, more about that will be down below.


Data are in many cases provided through the web APIs. The most straightforward definition of the web APIs is following.
Web API is the interface through which an application can communicate with that server. Then fragmentation of the web APIs is so significant that many existing solutions support only one type
and does not allow to get responses from multiple APIs simultaneously. More detailed
information about web API is in Chapter 3.


With all that data provided by the web APIs, there is a question - “How to render that data as options efficiently?”.  Because rendering large lists, is operation can be time-consuming and can affect the performance of an application significantly. Performance is one of the most critical aspects of the modern applications. Based on the study \parencite{Tolarance_study} the delay of 2 seconds is the limit where a response to simple commands becomes unacceptable to users.

%----------------------------------------------------------------------------------------
%	SECTION 2
%----------------------------------------------------------------------------------------

\section{Current situation}

UI developers often need to use a tree select input in their applications. The Problem is that default HTML select input is not usable in many cases. Also, most current solutions cannot handle complex data (e.g., graph data) or cannot render large data lists efficiently. Moreover, if the developers finally find a solution with functionality that they need, the solution does not fit by design into their application. So in many cases, they have to develop their component for that specific problem.


%----------------------------------------------------------------------------------------
%	SECTION 3
%----------------------------------------------------------------------------------------

\section{Thesis goal}

The primary goal of this bachelor thesis was to create a UI component that will support all features of the previous component such as rendering data as a tree, support of linked data, and support of graph type data structure. Also, the component should provide several new features. These features are:
\begin{itemize}
\item selecting or multi-selecting options
\item simplicity of use and integration with other applications
\item flexibility (e.g. customization, different types of datasets)
\item re-usability
\item good performance with large datasets
\item creating new options
\end{itemize}


%----------------------------------------------------------------------------------------
%	SECTION 4
%----------------------------------------------------------------------------------------

\section{Outline}

This thesis is divided into two main parts - research part and implementation part. In the research part, the primary focus will be on the technology and work related to this problem. The second part will describe the implementation of the component, logical structure, and behavior. So the second chapter will be dedicated to the used technology, and to the related problems. Especially the Linked Data and the concept of the semantic web. Then we will have a look at the APIs that provide these data, their construction and how they behave. In the next part, we will look at the technical aspects of the component such. Then the focus will be on the component itself. A chapter about implementation will describe individual parts or sub-components and at the end of that chapter algorithms for filtering and rendering will be described. In the end, there will be a comparison of other solutions and benchmark. The last chapter will summarise the results and describe future steps.


%----------------------------------------------------------------------------------------
%	SECTION 5
%----------------------------------------------------------------------------------------

\section{Term definition}

In this theses will be often used a several terms. To avoid any misunderstanding this table shows description of each term.

\begin{center}
    \begin{longtable}{ | l | p{10cm} | }
    \caption{Definition of terms used in this thesis} \label{tab:definition} \\
    \hline 
    \multicolumn{1}{|c|}{\textbf{Term}} & 
    \multicolumn{1}{c|}{\textbf{Description}}
	\endfirsthead
    
    \hline 
    select input & HTML element represented as <input type='select'/> \\ \hline
    dropdown menu & HTML elements displayed under select input. This element contains individual options \\ \hline
    option & Part of the dropdown men. Option represent one data object. Example of data object: \{value: '123', label: 'option label'\} . \\ \hline
    option provider & Web API providing data. \\ \hline
    search & Process of filtering and rendering results of filter subprocess in form of dropdown menu with options. \\ \hline
    \end{longtable}
\end{center}

%----------------------------------------------------------------------------------------